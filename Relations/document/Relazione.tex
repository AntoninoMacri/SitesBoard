\documentclass[12pt]{article}


\usepackage[english,italian]{babel} %lingue utilizzabili
\usepackage{graphicx}
\usepackage{fancyhdr}


\usepackage[utf8]{inputenc} %codifica e input

\usepackage{hyperref} %indice linkato ai paragrafi di riferimento
\hypersetup{
  colorlinks,
  citecolor=gray,
  filecolor=brown,
  linkcolor=black,
  urlcolor=cyan
}

\usepackage{lastpage} %per sapere il numero di pagine totali

\usepackage{parcolumns} %testo affiancato in colonna

\usepackage{listings} %Per inserire codice
\usepackage[usenames]{color} %Per permettere la colorazione dei caratteri 

\usepackage[a4paper,top=3cm,bottom=2cm,left=2cm,right=2cm] {geometry} %imposta pagina formato A4 e setta margini a piacimento
\usepackage[pdftex]{lscape} %per settare pagine in landscape

\renewcommand{\familydefault}{\sfdefault}   %mette tutto il carattere con uno stile ''grazioso'' <-----------------------------------molto bello

\definecolor{editorGray}{rgb}{0.95, 0.95, 0.95}
\definecolor{editorOrange}{rgb}{1, 0.5, 0} % #FF7F00 -> rgb(239, 169, 0)
\definecolor{editorGreen}{rgb}{0, 0.6, 0} % #007C00 -> rgb(0, 124, 0)




%personalizzazione classe per il codice SQL
\lstnewenvironment{sql}[1][] 
{\lstset{basicstyle=\scriptstyle \ttfamily, columns=fullflexible, keywordstyle=\color{blue}\bfseries, ndkeywordstyle=\color{blue}\bfseries , ndkeywords={references}, numberstyle=\color{red}, commentstyle=\color{editorGreen}, showstringspaces=false, stringstyle=\color{editorOrange},
language=SQL, basicstyle=\small,
numbers=left, numberstyle=\tiny,
tabsize=2, stepnumber=10, numbersep=5pt, breaklines=true, frame=single, rulecolor=\color{black}, #1}}
{\lstset {language=SQL,morekeywords={INT,CHAR,varchar,smallint,numeric}}}{}


%personalizzazione classe per il codice HTML
\lstnewenvironment{html}[1][] 
{\lstset{basicstyle=\scriptstyle \ttfamily, columns=fullflexible,
keywordstyle=\color{blue}\bfseries, ndkeywords={content,=,charset=, id=, width=, height=},	 ndkeywordstyle=\color{editorGreen}\bfseries , numberstyle=\color{red} commentstyle=\color{red}, showstringspaces=false, stringstyle=\color{editorOrange},
language=HTML, basicstyle=\small,
numbers=left, numberstyle=\tiny,
tabsize=2, stepnumber=10, numbersep=5pt, breaklines=true, frame=single, rulecolor=\color{black}, #1}}{}

%personalizzazione classe per il codice PERL
\lstnewenvironment{perl}[1][] 
{\lstset{basicstyle=\scriptstyle \ttfamily, columns=fullflexible,
keywordstyle=\color{blue}\bfseries, ndkeywords={content,=,charset=, id=, width=, height=},	 ndkeywordstyle=\color{editorGreen}\bfseries , numberstyle=\color{red} commentstyle=\color{red}, showstringspaces=false, stringstyle=\color{editorGreen},
language=PHP, basicstyle=\small,
numbers=left, numberstyle=\tiny,
tabsize=2, stepnumber=10, numbersep=5pt, breaklines=true, frame=single, rulecolor=\color{black}, #1}}{}





\begin{document}

\begin{figure}
\centering
\includegraphics[angle=0,scale=.30]{logo.png}
\end{figure} 

\title{ \textbf{{\Huge SitesBoard}}\vspace{2cm} \\ {\Huge Tecnologie Web \\ Relazione del progetto} \\ {\Large Anno accademico 2015/2016} }
%\date{}
 
\author{
\begin{tabular}{r|l}
\textbf{Componenti} & Fasolato Francesco\\
&Macrì Antonino\\
&Rigoni Davide\\
&Zecchin Giacomo
\end{tabular}\vspace{0.5cm} \\
	Indirizzo Web: \url{http://tecnologie-web.studenti.math.unipd.it/tecweb/~drigoni}\vspace{0.3cm} \\
		Email responsabile: \href{mailto:davide.rigoni.2@studenti.unipd.it}{davide.rigoni.2@studenti.unipd.it} 
}

\maketitle
\thispagestyle{empty}


\newpage


%da qui stile intestazione e piè di pagina cosi definiti
\pagestyle{fancy}
\lhead{}
\lfoot{\emph{\large SitesBoard}} 
\cfoot{}
\rfoot{Pagina: \thepage\ di \pageref{LastPage}}
\renewcommand{\footrulewidth}{0.5pt}
%%%%%%%%%%%%%%%%%%%%%%%%%

\tableofcontents %CREA INDICE



\newpage

\section{Abstract}
Come suggerisce il nome, SitesBoard è una bacheca digitale contenente inserzioni che offrono opportunità lavorative concernenti la realizzazione di siti web.\\
SitesBoard permette agli utenti registrati al suo interno di inserire nuovi annunci oppure accettare quelli proposti da altri utenti.\\\\
Ogni utente ha un’area personale, nella quale sono presenti i propri dati profilo.\\
Da quest’area può accedere alla pagina degli annunci che ha inserito o a quella degli annunci che ha accettato.\\\\
Per ogni annuncio proposto, è visibile la lista degli utenti che hanno dato la loro disponibilità. Una volta scaduto, spetta a chi l’ha pubblicato contattare il candidato che ritiene essere più adatto per quel lavoro.\\
Un utente ha anche la possibilità di cancellare un annuncio prima della data di scadenza.\\\\
Gli annunci sono divisi in categorie per aiutare gli sviluppatori a trovare quelli più adatti alle loro abilità. Le categorie sono: E-commerce, Forum, Social, Personali, Aziendali e Blog.\\
In ogni categoria gli annunci appaiono all’utente ordinati secondo la data di inserimento.\\
%\selectlanguage{italian}	%%%%inserirlo causerebbe errore in compilazione per lettere accentate-->non risolto


\section{Utenti destinatari}
Il nostro sito si rivolge per lo più ad un’utenza formata da persone abbastanza abili nell’uso di linguaggi di programmazione. \\
Una piccola parte di utenti sarà formata anche da persone che non hanno molte conoscenze informatiche e che, proprio per questo, usufruiranno del nostro servizio digitale per contattare qualcuno in grado di realizzare un sito web.
\newpage

\section{Progettazione}
	\subsection{Struttura}
		\subsubsection{Organizzazione cartelle}
		
		I file che compongono il sito sono organizzati su 3 cartelle:

		\begin{itemize}

			\item \textbf{public\textunderscore html}: contiene le pagine statiche, gli stili, il codice javaScript e le immagini contenute nel sito;
			\item \textbf{data}: contiene i files accessibili solo internamente al server, quindi i file xml, i fogli di trasformazione e la definizione dello schema;
			\item \textbf{cgi-bin}: contiene gli script cgi che generano dinamicamente le pagine del sito, uno script in Perl per il controllo e la gestione della sessione, uno contenente le funzioni utilizzate da più pagine ed alcuni script di controllo e gestione delle altre pagine.

		\end{itemize}


		\subsubsection{Definizione}
	
		Il sito visualizzabile da un utente comune non registrato/loggato si compone di 10 pagine:
		
		\begin{itemize}
			\item \textbf{home.cgi}: È la homepage del sito; mostra gli annunci di tutti gli utenti in ordine di data.
			\item \textbf{pagine suddivise per le diverse categorie di annunci}: Le pagine eCommerce.cgi, forum.cgi, social.cgi, personali.cgi, aziendali.cgi e blog.cgi mostrano gli annunci della loro tipologia.
			\item \textbf{registration.cgi}: È la pagina che contiene la form per la registrazione al sito di SitesBoard.
			\item \textbf{login.cgi}: È la pagina che contiene la form per eseguire la login al sito.
			\item \textbf{pass\textunderscore recovery.html}: È la pagina che contiene la form per il recupero della password utente eventualmente dimenticata.\\
		\end{itemize}
		Le pagine alle quali hanno accesso solamente gli utenti che hanno effettuato il login sono 7:
		
		\begin{itemize}
			\item \textbf{profile.cgi}: È la pagina che mostra i dati personali dell'utente che ha effettuato il login.
			\item \textbf{profileChange.cgi}: È la pagina che contiene la form per la modifica dei dati personali dell'utente.
			\item \textbf{addInsertions.cgi}: È la pagina che contiene la form per l'inserimento di un nuovo annuncio.
			\item \textbf{showInsertions.cgi}: È la pagina che mostra gli annunci che ha inserito l'utente.
			\item \textbf{acceptedInsertions.cgi}: È la pagina che mostra gli annunci accettati dall'utente.
			\item \textbf{userProfile.cgi}: È la pagina che mostra i dati di un altro utente registrato al sito.
			\item \textbf{insertion.cgi}: È la pagina che mostra un singolo annuncio inserito da un altro utente.
		\end{itemize}
		
	
		\subsubsection{Database}
		Il database è stato realizzato utilizzando un unico file chiamato “database.xml”.\\\\
		Avremmo potuto infatti dividere il nostro database in 2 file distinti, ad esempio “utenti.xml” e “annunci.xml”.
		La nostra scelta è stata dovuta alla maggiore praticità nel gestire un singolo file e al fatto che interrogare più database in quasi tutte le query sarebbe stato troppo dispendioso.\\
		Il file è stato strutturato seguendo il modello a bambole russe.
		Le motivazioni che hanno portato a questa scelta sono dovute alle caratteristiche di questo modello:
		\begin{itemize}
			\item Semplifica la leggibilità della struttura poiché rispecchia la struttura del documento XML.
			\item Ogni tipo complesso ha una struttura diversa dagli altri tipi complessi perciò ogni elemento è definito localmente.\\
		\end{itemize}
		La struttura del documento XML è stata definita con XML Schema poiché esso fornisce una descrizione più dettagliata di quella che avrebbe potuto definire un file DTD.

	\subsection{Presentazione}
	
	Il sito è impostato secondo un layout ibrido a 3 pannelli.\bigskip
	
	Per una maggiore usabilità sono presenti due differenti visualizzazioni del sito, a seconda del dispositivo utilizzato per la navigazione.\\
	È stato impostato un punto di rottura a 800px di larghezza al di sotto del quale si passa da una visualizzazione desktop ad una adatta per dispositivi mobile o tablet con piccoli schermi.\bigskip
	
	A sinistra c'è il menù di navigazione che permette all’utente di raggiungere varie sezioni del sito in base al suo interesse, in alto l'header e la breadcrumb, nella zona centrale il corpo del contenuto ed infine in basso il footer.
Per quanto riguarda il layout ridotto, esso si sviluppa in verticale. Dall’alto verso il basso si trovano in ordine header, breadcrumb, il menù di navigazione, il corpo e il footer.\bigskip
		
	La grandezza del font dell'intero corpo è stata impostata in ems invece dei pixel.
	Essendo una misura relativa, gli ems fanno si che anche il testo si adatti alla dimensione dello schermo visualizzato. Inoltre se vogliamo cambiare successivamente la dimensione del font, mentre con i pixel dobbiamo modificarlo in ogni parte del sito web, con gli ems è sufficiente cambiarlo nel body ed una sola volta poiché tutti gli altri font si adatteranno proporzionalmente.\\
	Dobbiamo anche considerare la non scalabilità dei pixel in IE nei browser più vecchi. Se un un utente ha bisogno di ingrandire le dimensioni di tutto il sito il font settato in pixel rimane della stessa dimensione creando gravi problemi di accessibilità poiché molti utenti con problemi alla vista non riuscirebbero ad accedere ai contenuti.
	Anche gli ems però presentano un piccolo svantaggio che è quello di dover considerare il contesto in cui si trovano se vogliamo mantenere un testo reattivo.\\ Visto che però il contenuto di SitesBoard è rappresentato quasi unicamente da testo (sono presenti solo 3 loghi nell'header) il problema di gestire immagini o altri contenuti grafici diventa irrilevante.
	
	Il layout così impostato diventa molto fluido, accessibile ed adattabile ad ogni dispositivo di navigazione e ciò sta alla base di un responsive layout.
	
	\newpage
	
	
	\begin{figure}
	\includegraphics[angle=0,scale=.7]{layout1.png}
	\includegraphics[angle=0,scale=.7]{layout2.png}
	\hspace{1.5in}
	%\caption{layout}
	\end{figure}
	
	\subsection{Javascript}
	
	Il linguaggio javascript è stato utilizzato principalmente per i controlli	
	
	\subsection{Perl}
	
	
	\subsection{XSLT}


\section{Accessibilità}
[..]
	\subsection{Struttura}
	\begin{itemize}
	\item Le pagine Web hanno tutte dei titoli che ne descrivono l’argomento o la finalità.
	\item Per ogni immagine presente è stato associato l'attributo “alt” in modo da fornire un’alternativa testuale per contenuti non testuali.
	\item Uso di attributo “xml:lang” per identificare le parole in lingua straniera, in modo che uno screen reader possa leggerle correttamente.
	\item Uso di breadcrumb (path) per permettere all’utente di sapere sempre dove si trova nel sito.
	\item Le form son tutte divise in fieldset ed anche se non sono esteticamente belle le
	\item Note: The maxlength attribute of the textarea tag is not supported in Internet Explorer 9 and earlier versions, or in Opera 12 and earlier versions.
	\end{itemize}

	\subsection{Presentazione}

	\subsection{Comportamento}

	\subsection{Test effettuati}
		\subsubsection{Conformità standard WCAG}
		\subsubsection{Contrato testo e sfondo}
		\subsubsection{Screen reader}		
		\subsubsection{Disturbi visivi}
		
\section{Usabilità}
Particolare attenzione è stata posta all’usabilità del sito e si è cercato dunque di rispettare il più possibile le sue raccomandazioni:

\begin{itemize}
\item \textbf{Le sei W}: La home page del sito risponde alle seguenti domande:

\begin{itemize}
\item Where: Il breve testo presente nelle pagina di apertura illustra in maniera esaustiva ciò che il sito si propone di rappresentare, ovvero una bacheca per proporre e trovare persone affidabili per la realizzazione di siti;
\item Who: Un utente che accede per la prima volta ad un sito sente la necessità di conoscere chi o quale entità il esso rappresenta. Abbiamo quindi posizionato, come di consueto, in alto a sinistra il logo del sito seguito alla sua destra dal nome e dallo slogan. Il logo è cliccabile e porta sempre l'utente nella home page del sito;
\item When: Nelle sezioni degli annunci quest'ultimi sono visualizzati per data, facendo si che l'ultimo inserito si trovi sempre in cima alla pagina e che quindi sia sempre il più visibile;
\item How: I menù posti sulla sinistra della pagina sono suddivisi per funzione e permettono all'utente di visualizzare tutte le sezioni del sito;
\item What: Un utente, quando accede alla home page riesce a farsi un’idea del sito visualizzando lo slogan e una descrizione del sito.
\end{itemize}

\item \textbf{Navbar}:

\item \textbf{Breadcrumbs}:

\item \textbf{Link}
Inserito a:hover{font-variant: small-caps} per risaltare i link e aumentare accessibilità.


\end{itemize}

		
\section{Validazione}
	\subsection{XHTML}
	\subsection{CSS}
	\subsection{XML e XSD}

\section{xmlSchema}



\section{Ruoli}

	
%per inserire un'immagine
%\begin{landscape}
%\begin{figure}
%\subsection{Schema E-R iniziale del database}
%\centering
%\includegraphics[angle=0,scale=.40]{projBasi.jpg}
%\hspace{1in}
%\label{schema}
%\caption{schema E-R}
%\end{figure}
%\end{landscape}


\newpage












\end{document}

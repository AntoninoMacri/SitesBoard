\documentclass[12pt]{article}


\usepackage[english,italian]{babel} %lingue utilizzabili
\usepackage{graphicx}
\usepackage{fancyhdr}


\usepackage[utf8]{inputenc} %codifica e input

\usepackage{hyperref} %indice linkato ai paragrafi di riferimento
\hypersetup{
  colorlinks,
  citecolor=gray,
  filecolor=brown,
  linkcolor=black,
  urlcolor=cyan
}

\usepackage{lastpage} %per sapere il numero di pagine totali

\usepackage{parcolumns} %testo affiancato in colonna

\usepackage{listings} %Per inserire codice
\usepackage[usenames]{color} %Per permettere la colorazione dei caratteri 

\usepackage[a4paper,top=3cm,bottom=2cm,left=2cm,right=2cm] {geometry} %imposta pagina formato A4 e setta margini a piacimento
\usepackage[pdftex]{lscape} %per settare pagine in landscape

\renewcommand{\familydefault}{\sfdefault}   %mette tutto il carattere con uno stile ''grazioso'' <-----------------------------------molto bello

\definecolor{editorGray}{rgb}{0.95, 0.95, 0.95}
\definecolor{editorOrange}{rgb}{1, 0.5, 0} % #FF7F00 -> rgb(239, 169, 0)
\definecolor{editorGreen}{rgb}{0, 0.6, 0} % #007C00 -> rgb(0, 124, 0)




%personalizzazione classe per il codice SQL
\lstnewenvironment{sql}[1][] 
{\lstset{basicstyle=\scriptstyle \ttfamily, columns=fullflexible, keywordstyle=\color{blue}\bfseries, ndkeywordstyle=\color{blue}\bfseries , ndkeywords={references}, numberstyle=\color{red}, commentstyle=\color{editorGreen}, showstringspaces=false, stringstyle=\color{editorOrange},
language=SQL, basicstyle=\small,
numbers=left, numberstyle=\tiny,
tabsize=2, stepnumber=10, numbersep=5pt, breaklines=true, frame=single, rulecolor=\color{black}, #1}}
{\lstset {language=SQL,morekeywords={INT,CHAR,varchar,smallint,numeric}}}{}


%personalizzazione classe per il codice HTML
\lstnewenvironment{html}[1][] 
{\lstset{basicstyle=\scriptstyle \ttfamily, columns=fullflexible,
keywordstyle=\color{blue}\bfseries, ndkeywords={content,=,charset=, id=, width=, height=},	 ndkeywordstyle=\color{editorGreen}\bfseries , numberstyle=\color{red} commentstyle=\color{red}, showstringspaces=false, stringstyle=\color{editorOrange},
language=HTML, basicstyle=\small,
numbers=left, numberstyle=\tiny,
tabsize=2, stepnumber=10, numbersep=5pt, breaklines=true, frame=single, rulecolor=\color{black}, #1}}{}

%personalizzazione classe per il codice PERL
\lstnewenvironment{perl}[1][] 
{\lstset{basicstyle=\scriptstyle \ttfamily, columns=fullflexible,
keywordstyle=\color{blue}\bfseries, ndkeywords={content,=,charset=, id=, width=, height=},	 ndkeywordstyle=\color{editorGreen}\bfseries , numberstyle=\color{red} commentstyle=\color{red}, showstringspaces=false, stringstyle=\color{editorGreen},
language=PHP, basicstyle=\small,
numbers=left, numberstyle=\tiny,
tabsize=2, stepnumber=10, numbersep=5pt, breaklines=true, frame=single, rulecolor=\color{black}, #1}}{}





\begin{document}

\begin{figure}
\centering
\includegraphics[angle=0,scale=.30]{Logo.png}
\end{figure} 

\title{ \textbf{{\Huge SitesBoard}}\vspace{2cm} \\ {\Huge Tecnologie Web \\ Relazione del progetto} \\ {\Large Anno accademico 2015/2016} }
%\date{}
 
\author{
\begin{tabular}{r|l}
\textbf{Componenti} & Fasolato Francesco\\
&Macrì Antonino\\
&Rigoni Davide\\
&Zecchin Giacomo
\end{tabular}\vspace{0.5cm} \\
	Indirizzo Web: \url{http://tecnologie-web.studenti.math.unipd.it/tecweb/~drigoni}\vspace{0.3cm} \\
		Email responsabile: \href{mailto:davide.rigoni.2@studenti.unipd.it}{davide.rigoni.2@studenti.unipd.it} 
}

\maketitle
\thispagestyle{empty}


\newpage


%da qui stile intestazione e piè di pagina cosi definiti
\pagestyle{fancy}
\lhead{}
\lfoot{\emph{\large SitesBoard}} 
\cfoot{}
\rfoot{Pagina: \thepage\ di \pageref{LastPage}}
\renewcommand{\footrulewidth}{0.5pt}
%%%%%%%%%%%%%%%%%%%%%%%%%

\tableofcontents %CREA INDICE



\newpage

\section{Abstract}
Come suggerisce il nome, SitesBoard è una bacheca digitale contenente inserzioni che offrono opportunità lavorative concernenti la realizzazione di siti web.\\
SitesBoard permette agli utenti registrati al suo interno di inserire nuovi annunci oppure accettare quelli proposti da altri utenti.\\\\
Ogni utente ha un’area personale, nella quale sono presenti i propri dati profilo.\\
Da quest’area può accedere alla pagina degli annunci che ha inserito o a quella degli annunci che ha accettato.\\\\
Per ogni annuncio proposto, è visibile la lista degli utenti che hanno dato la loro disponibilità. Una volta scaduto, spetta a chi l’ha pubblicato contattare il candidato che ritiene essere più adatto per quel lavoro.\\
Un utente ha anche la possibilità di cancellare un annuncio prima della data di scadenza.\\\\
Gli annunci sono divisi in categorie per aiutare gli sviluppatori a trovare quelli più adatti alle loro abilità. Le categorie sono: E-commerce, Forum, Social, Personali, Aziendali e Blog.\\
In ogni categoria gli annunci appaiono all’utente ordinati secondo la data di inserimento.\\
%\selectlanguage{italian}	%%%%inserirlo causerebbe errore in compilazione per lettere accentate-->non risolto


\section{Utenti destinatari}
Il nostro sito si rivolge per lo più ad un’utenza formata da persone abbastanza abili nell’uso di linguaggi di programmazione. \\
Una piccola parte di utenti sarà formata anche da persone che non hanno molte conoscenze informatiche e che, proprio per questo, usufruiranno del nostro servizio per contattare qualcuno in grado di realizzare un sito web.
\newpage

\section{Progettazione}
	\subsection{Struttura}
		\subsubsection{Organizzazione cartelle}
		
I file che compongono il sito sono organizzati su 3 cartelle:

\begin{itemize}

\item \textbf{public\textunderscore html}: che contiene le pagine statiche, gli stili, il codice javaScript e le immagini contenute nel sito;
\item \textbf{data}: che contiene i documenti accessibili solo internamente al server, quindi i file xml, i fogli di trasformazione, la definizione degli schemi e le pagine html private;
\item \textbf{cgi-bin}: che contiene gli script scritti in Perl.

\end{itemize}


	\subsubsection{Definizione}
	\subsubsection{Database}
	Il database è stato realizzato utilizzando un unico file chiamato “database.xml”.\\\\
	La nostra scelta è stata dovuta alla maggiore praticità nel gestire un singolo file e al fatto che interrogare più database in quasi tutte le query sarebbe stato troppo dispendioso.\\
	Avremmo potuto infatti dividere il nostro database in 2 file distinti, ad esempio “utenti.xml” e “annunci.xml”. Il problema della doppia interrogazione, però, si sarebbe manifestato visualizzando un annuncio, nel quale è presente il nome dell’utente che l’ha inserito, con il relativo link alla pagina del profilo personale.\\\\
	Il file è stato strutturato seguendo il modello a bambole russe, scelto per la maggiore leggibilità, (vedi slide)\\
	
	Abbiamo realizzato la parte… con xml-schema, poiché fornisce

	\subsection{Presentazione}
	\subsection{Javascript}
	\subsection{Perl}
	\subsection{XSLT}


\section{Accessibilità}
[..]
	\subsection{Struttura}
	\begin{itemize}
	\item Le pagine Web hanno tutte dei titoli che ne descrivono l’argomento o la finalità.
	\item Per ogni immagine presente è stato associato l'attributo “alt” in modo da fornire un’alternativa testuale per contenuti non testuali.
	\item Uso di attributo “xml:lang” per identificare le parole in lingua straniera, in modo che uno screen reader possa leggerle correttamente.
	\item Uso di breadcrumb (path) per permettere all’utente di sapere sempre dove si trova nel sito.
	\item Le form son tutte divise in fieldset ed anche se non sono esteticamente belle le
	\end{itemize}

	\subsection{Presentazione}

	\subsection{Comportamento}

	\subsection{Test effettuati}
		\subsubsection{Conformità standard WCAG}
		\subsubsection{Contrato testo e sfondo}
		\subsubsection{Screen reader}		
		\subsubsection{Disturbi visivi}
		
\section{Usabilità}
Particolare attenzione è stata posta all’usabilità del sito e si è cercato dunque di rispettare il più possibile le sue raccomandazioni:

\begin{itemize}
\item \textbf{Le sei W}: La home page del sito risponde alle seguenti domande:

\begin{itemize}
\item Where: Il breve testo presente nelle pagina di apertura illustra in maniera esaustiva ciò che il sito si propone di rappresentare, ovvero una bacheca per proporre e trovare persone affidabili per la realizzazione di siti;
\item Who: Un utente che accede per la prima volta ad un sito sente la necessità di conoscere chi o quale entità il esso rappresenta. Abbiamo quindi posizionato, come di consueto, in alto a sinistra il logo del sito seguito alla sua destra dal nome e dallo slogan. Il logo è cliccabile e porta sempre l'utente nella home page del sito;
\item When: Nelle sezioni degli annunci quest'ultimi sono visualizzati per data, facendo si che l'ultimo inserito si trovi sempre in cima alla pagina e che quindi sia sempre il più visibile;
\item How: I menù posti sulla sinistra della pagina sono suddivisi per funzione e permettono all'utente di visualizzare tutte le sezioni del sito;
\item What: Un utente, quando accede alla home page riesce a farsi un’idea del sito visualizzando lo slogan e una descrizione del sito.
\end{itemize}

\item \textbf{Navbar}:

\item \textbf{Breadcrumbs}:

\item \textbf{Link}

\item \textbf{Foto}


\end{itemize}

		
\section{Validazione}
	\subsection{XHTML}
	\subsection{CSS}
	\subsection{XML e XSD}

\section{xmlSchema}

pregi: (guarda lucidi xml sul sito)
è il più semplice da comprendere\\
difetti: non è possibile riutilizzare le strutture precedentemente definite e il contenuto è opaco---->nel nostro caso tutti i tipi complessi nn si ripetono quindi il problema di riscrivere la stessa definizione di una struttura non si pone.




\section{Ruoli}

	
%per inserire un'immagine
%\begin{landscape}
%\begin{figure}
%\subsection{Schema E-R iniziale del database}
%\centering
%\includegraphics[angle=0,scale=.40]{projBasi.jpg}
%\hspace{1in}
%\label{schema}
%\caption{schema E-R}
%\end{figure}
%\end{landscape}


\newpage












\end{document}
